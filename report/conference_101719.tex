\documentclass[conference]{IEEEtran}
\IEEEoverridecommandlockouts
% The preceding line is only needed to identify funding in the first footnote. If that is unneeded, please comment it out.
\usepackage{cite}
\usepackage{amsmath,amssymb,amsfonts}
\usepackage{algorithmic}
\usepackage{graphicx}
\usepackage{textcomp}
\usepackage{xcolor}
\def\BibTeX{{\rm B\kern-.05em{\sc i\kern-.025em b}\kern-.08em
    T\kern-.1667em\lower.7ex\hbox{E}\kern-.125emX}}
\begin{document}

\title{Biomarkers in Breast Cancer Diagnosis: an Exploratory Data Analysis\\}

\author{\IEEEauthorblockN{Saulo Mendes de Melo}
\IEEEauthorblockA{\textit{Applied Computational Intelligence} \\
\textit{PPGETI}\\
Fortaleza, Brazil \\
saulomelo96@hotmail.com}
}

\maketitle

\begin{abstract}
This document is a model and instructions for \LaTeX.
This and the IEEEtran.cls file define the components of your paper [title, text, heads, etc.]. *CRITICAL: Do Not Use Symbols, Special Characters, Footnotes, 
or Math in Paper Title or Abstract.
\end{abstract}

\begin{IEEEkeywords}
Breast cancer, PCA, biomarkers, Data analysis
\end{IEEEkeywords}

\section{Introduction}
Cancer is among the leading causes of death worldwide and every 
year its various types claim a great number of lives. Amongst 
women victims, breast cancer (BC) is the leading type, with as much 
as 322.000 deaths in the year 1990\cite{murray1997mortality}. A 
study published in 2019 analyzing the US population has found 
that approximately 13\% of women will be diagnosed with BC in their 
lifetime, and as much as 1 in 39 women will 
eventually succumb to it\cite{desantis2019breast}.

The incidence of BC among women increases with age. 
The probability of a diagnose for woman in the age of 20 is of 
0.1\%, and goes to 3.0\% by the age of 80. The rates of incidence 
and mortality also hold some relationship with ethnicity. The 
incidence rate is higher among whites (130.8 per 100.000), 
but for blacks, while incidence is lower the mortality is up 
to 40\% higher (28.4 per 100.00)\cite{desantis2019breast}. 
Other risk factors associated are late first birth (post 30), 
nulliparity, use of oral contraceptives and having first and/or 
second-degree relatives diagnosed\cite{vogel2018epidemiology}.

Although non-clinical factors provide a valuable insight, other 
studies have focused on the analysis of biological indicators more 
tipically related to medical assessment. Dalamaga\cite{dalamaga2014resistin} 
has analyzed Resistin as biomarker, and it's links to obesity and 
cancer. Crisóstomo et al.\cite{crisostomo2016hyperresistinemia} provided 
a study of biomarkers in the context of BC, where groups were 
separated for both obese and non-obese control and patients, and an 
extensive statistical analysis was made with indicators such as 
Glucose, Insulin, Resistin, among others. A set metabolic characteristics 
was found in obese women with BC, which includes glucoes, insulin disorders 
and other anomalies.

The present work aims to provide an analysis of biomarkers on a clinical 
dataset, evaluate how correlated they might be with BC detection in 
patients and the possibilities of dimensionality reduction through 
the use of Principal Component Analysis.


\bibliographystyle{IEEEtran}
\bibliography{citations}

\end{document}
